\begin{frame}{Requisitos}
    \begin{itemize}
        \item Para poder utilizar \LaTeX basicamente você precisa de:
        \begin{enumerate}
            \item Editor de texto
            \item Compilador
        \end{enumerate}

        \item O arquivo fonte de um texto em \LaTeX é basicamente:
        \begin{itemize}
            \item texto
            \item \textit{tags}
        \end{itemize}
    \item No final, o arquivo gerado é (normalmente): .ps ou .pdf
    \item Porém, existem ferramentas que auxiliam nesse processo
    \end{itemize}
\end{frame}

\begin{frame}{Editor \& Compilador}
    \begin{itemize}
        \item O arquivo fonte do texto \LaTeX pode ser escritos em um editor de texto simples ou específico
        
        \item Os específicos trazem algumas vantagens como:
        \begin{itemize}
            \item compilação automática
            \item \textit{highlight}
            \item \textit{templates}
            \item \textit{autocomplete}
            \item pré-visualização instantânea
            \item dentre outras
        \end{itemize}
    \end{itemize}
\end{frame}

\begin{frame}{Ferramentas de auxílio}
    
    \begin{itemize}
        \item Alguns editores específicos para \LaTeX são:
        \begin{itemize}
            \item Desktop:
            \begin{itemize}
                \item TexStudio
                \item TeXnicCenter
                \item TexMaker
                \item Kile
            \end{itemize}
            \item Web:
                \begin{itemize}
                \item Share\LaTeX
                \item Papeeria
                \item Overleaf
                \item Datazar
            \end{itemize}
        \end{itemize}
        \item Os editores Desktop podem ser mais personalizados e eficientes 
        \item Já os editores Web resolvem a configuração do ambiente \LaTeX
         
    \end{itemize}
\end{frame}

%Linux
\begin{frame}{Instalação em Ambientes Desktop}
    \begin{itemize}
    \item Linux
    \begin{itemize}
        \item Vai depender muito da distribuição
        \item Geralmente, busca pelo pacote \textit{texlive} no gerenciador de pacotes
        \item Baixar e instalar o editor
        \item Sugestão de Editores: Kile ou TeXStudio 
    \end{itemize}
    \end{itemize}
\end{frame}
%Windows
\begin{frame}{Instalação em Ambientes Desktop}
    \begin{itemize}
        \item Windows
        \begin{itemize}
            \item Escolher um Editor!
            \item Baixar e instalar o GhostScript
            \item Baixar e instalar o Ghostview ou GSView
            \item Baixar e instalar o Miktex (\TeX e gerenciador de pacotes
            para Windows)
            \item Finalmente... baixar e instalar o editor
            \item Sugestão de Editores: TeXnicCenter ou TeXStudio            
        \end{itemize}
    \end{itemize}
\end{frame}