\begin{frame}{Primeiro, a briga da pronuncia...}
    \begin{itemize}
        \item  O nome \TeX é composto por três letras gregas:
        \begin{itemize}
            \item $\tau$ (tau)
            \item $\epsilon$ (épsilon)
            \item $\chi$ (chi, pronunciado \textit{qui})
        \end{itemize}
        \item Daí vem $\tau$$\epsilon$$\chi$, ou téc
        \item Finalmente, \LaTeX, ou \textit{latéc} (em inglês ficaria lei-téc)
        \item Porém, não é bem isso que acontece... então fique a vontade para chamar como se sentir mais confortável
    \end{itemize}
    

\end{frame}
%

\begin{frame}{O que é o \LaTeX?}
    \begin{itemize}
        \item Em poucas palavras:  \LaTeX é um sistema tipográfico de alta qualidade que provê funcionalidades focadas na produção de textos
        \item  Pode ser utilizada para redação de qualquer documento: desde uma simples carta até livros completos, desde trabalhos de faculdade até artigos científicos, desde relatórios até apresentações...
    \end{itemize}
\end{frame}
%

\begin{frame}{E para que serve?}
    \begin{itemize}
        \item O \LaTeX é um grande aliado para pessoas que precisam produzir textos acadêmicos e científicos
        \item É geralmente utilizado principalmente por profissionais do meio acadêmico das áreas de Ciências Exatas
        \item Ideal para qualquer pessoa que queria otimizar a produção de textos e economizar tempo
        \item Facilita diversas atividades comuns na construção de um texto, tais como, uso de fórmulas e equações matemáticas; criação de tabelas e imagens; e gerência das referências e citações
    \end{itemize}
\end{frame}

\begin{frame}{Contexto Histórico}
    \begin{description}
        \item[1977] Donald E. Knuth criou um programa \TeX com linguagem própria para processar textos e fórmulas matemáticas eletronicamente com o objetivo de aumentar a qualidade de impressão naquela época
        \item[1982] Lançada a primeira versão estável do \TeX
        \item[1985] Leslie Lamport criou um conjunto de macros chamada \LaTeX para simplificar o uso do \TeX
        \item[]
        \item[] Oficial do projeto: \textit{\url{https://www.latex-project.org/}}
    \end{description}
    
    
\end{frame}
%

\begin{frame}{Mudando a Perspectiva}
    \begin{itemize}
        \item Processadores WYSIWYG (\textbf{W}hat \textbf{Y}ou \textbf{S}ee \textbf{I}s \textbf{W}hat \textbf{Y}ou \textbf{G}et)
        \begin{itemize}
            \item Exemplos: LibreOffice, MS Word, Corel WordPerfect...
        \end{itemize}
        \item[]
        \item Processadores WYSIWYM (\textbf{W}hat \textbf{Y}ou \textbf{S}ee \textbf{I}s \textbf{W}hat \textbf{Y}ou \textbf{M}ean)
        \begin{itemize}
            \item Exemplos: HTML, QML, \LaTeX...
        \end{itemize}
    \end{itemize}
\end{frame}
%

\begin{frame}{Pontos Negativos}
    \begin{itemize}
        \item Curva de aprendizado
        \item Complexidade de criar novos \textit{layouts} ou alterá-los
        \item Configuração do ambiente as vezes pode se tornar complicada*
        \item Procedimento de processamento (compilação) não é trivial,\\ inclusive em alguns editores
    \end{itemize}
\end{frame}

\begin{frame}{Pontos Positivo}
    \begin{itemize}
        \item São muitos!
        \item Tanto o \LaTeX e o \TeX  são \textbf{Open Source}
        \item Existem diversas ferramentas auxiliares
        \item Permitem criar textos com alta qualidade tipográfica
        \item Os textos e seus elementos ficam com aparência profissional
        \item Diversos \textit{layouts} e \textit{templates} prontos
        \item Estruturas tipográficas complexas (bibliografia, tabela de
        conteúdo, citações) podem ser criadas facilmente
        \item Numeração e citação automáticas
        \item Evitam erro de \textit{layouts}
        
    \end{itemize}
\end{frame}
\begin{frame}{Pontos Positivo}
    \begin{itemize}
        \item Pacotes para gerar vários tipos de documentos
        \begin{itemize}
            \item Artigos
            \item Relatórios
            \item Livros
            \item Slides
            \item Poster
            \item Apresentação
        \end{itemize}
        
        \item Foco nos comandos e não na estrutura
        \item Economia de tempo e foco no mais importante (o texto)
        \item Troca de \textit{layout} simples, as vezes com uma \textbf{única} palavra
        \item Possibilidade de versionamento
        \item E muitos outros...
    \end{itemize}
\end{frame}
